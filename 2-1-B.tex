%% B. Propulsion

%% 1. Define the key subsystem requirements and functions (e.g., orbit
%% insertion, orbital manoeuvres, interplanetary transfer,
%% etc.). Note: consider only the propulsion system on-board of the
%% orbiter spacecraft; design of optional kick stages (if any) is not
%% required.

%% 2. Define and quantify the eventual forces to be overcome by the
%% propulsion system (gravity, drag, solar radiation etc.).

%% 3. Revise your estimation for the total v required by the mission
%% and, based on it, select the thruster(s) and the propellants to be
%% used (note: decide at this stage if a “low thrust” or a “high
%% thrust” strategy has to be adopted, and select your thrusters
%% accordingly).

%% 4. Evaluate the total propellant mass needed by the mission. Don’t
%% forget to consider sufficient margins!!

%% 5. List and characterize all the components of the propulsion
%% systems; don’t consider the propellant tank(s), that will be
%% studied under the structures subsystem. If possible, select a
%% commercially available complete propulsion system.

%% 6. Generate a sketch showing the number and the location of the
%% various propulsion components in the spacecraft. Include in the
%% sketch also the major characteristics of the components shown.
