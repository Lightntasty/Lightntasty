%% E. Power

%% 1. Define the key power subsystem requirements and, in particular,
%% the total amount of power required by the spacecraft during all the
%% mission phases.

\subsubsection{Key requirements.}

The key requirement for the power system is to provide adequate power
for the duration of the whole mission. Solar panels are used as the
power source since these are asked for in the project reader. Solar
array degradation must be taken to account to provide adequate power
until the end of mission. Batteries are necessary to provide power
during eclipses.

%% 2. Quantify the available solar energy, i.e. the solar intensity
%% and the direction of solar radiation, the duration of day time and
%% night time (eclipse) periods, the energy received from the planet.

\subsubsection{Available solar energy.}

The minimum solar intensity $I$ at Jupiter occurs at the apohelion and is
\begin{equation}
  I_{\mathrm{min}} = I_{\si{AU}} \left(\frac{\SI{1}{AU} (1 - e)}{a}\right)^2
  \approx \SI{45.7}{W/m^2}
\end{equation}
The mean motion $n$ in Europa orbit is
\begin{equation}
  n = \sqrt{\frac{\mu}{a}}
\end{equation}
Then, the duration of a full orbit is
\begin{equation}
  T = \frac{2 \pi}{n} \approx \SI{137}{min}
\end{equation}
The angular span of an eclipse is
\begin{equation}
  \theta = 2 \arcsin \frac{R}{R + h} \approx \ang{125}
\end{equation}
where $R$ is the radius of Europa and $h$ is the altitude of the
circular orbit. The duration of an eclipse is
\begin{equation}
  T_{\mathrm{e}} = \frac{\theta}{n} \approx \SI{47}{min}
\end{equation}
and the duration of a day is
\begin{equation}
  T_{\mathrm{d}} = \frac{2 \pi - \theta}{n} \approx \SI{89}{min}
\end{equation}

%% 3. Make a preliminary dimensioning of the solar arrays and, in
%% particular, a first estimation of their size (note: detailed design
%% of the solar arrays will be done during WP 3!!). You should
%% consider, for instance, whether you would like a fixed array or an
%% array that can be rotated over one or two axes to allow for ideal
%% lighting conditions.

%% 4. Based on your power requirements, evaluate the main
%% characteristic properties of the spacecraft batteries (e.g.,
%% specific energy, energy density, capacity, mass, size, etc.).

%% 5. Select the batteries and the power control components among
%% commercially existing models.

%% 6. Generate a block diagram showing the main subsystem components,
%% the power flow and the efficiencies of power conversion.
