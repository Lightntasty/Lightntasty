%% E. Power

The minimum solar intensity $I$ at Jupiter occurs at the apohelion and is
\begin{equation}
  I_{\mathrm{min}} = I_{\si{AU}} (\frac{\SI{1}{AU} (1 - e)}{a})^2
  \approx \SI{45.7}{W/m^2}
\end{equation}

%% 1. Define the key power subsystem requirements and, in particular,
%% the total amount of power required by the spacecraft during all the
%% mission phases.

%% 2. Quantify the available solar energy, i.e. the solar intensity
%% and the direction of solar radiation, the duration of day time and
%% night time (eclipse) periods, the energy received from the planet.

%% 3. Make a preliminary dimensioning of the solar arrays and, in
%% particular, a first estimation of their size (note: detailed design
%% of the solar arrays will be done during WP 3!!). You should
%% consider, for instance, whether you would like a fixed array or an
%% array that can be rotated over one or two axes to allow for ideal
%% lighting conditions.

%% 4. Based on your power requirements, evaluate the main
%% characteristic properties of the spacecraft batteries (e.g.,
%% specific energy, energy density, capacity, mass, size, etc.).

%% 5. Select the batteries and the power control components among
%% commercially existing models.

%% 6. Generate a block diagram showing the main subsystem components,
%% the power flow and the efficiencies of power conversion.
