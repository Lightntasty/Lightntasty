%% D. Structures and Mechanisms

%% 1. Define the key requirements for the structures and mechanisms
%% subsystem.

%% 2. Define all the loads acting on the spacecraft during its
%% operations (note: take into

%% account the launch loads that you have already estimated during WP
%% 1!!). Define the safety factors to be used (you can use, for
%% instance, the values provided by the relevant ESA ECSS standards,
%% see Ref. [8]).

%% 3. Preliminarily dimension and design the primary structure of your
%% spacecraft, also choosing adequate materials for it. Based on the
%% relevant requirements for the spacecraft and its components, decide
%% if the structure has to be designed for stiffness and/or strength
%% and/or internal pressure. Hint: for your analysis you may consider
%% the primary structure to resemble a hollow cylindrical beam.

%% 4. Based on the amount of propellant(s) needed and the required
%% storage pressure, dimension and design the propellant tank(s), also
%% choosing adequate materials (note: the tank(s) can eventually be
%% used as primary structural elements!). Whereas possible, select
%% commercially existing tanks.

%% 5. Preliminarily dimension and design the supporting structure of
%% the solar arrays of your spacecraft (i.e., the mechanical structure
%% on which the array is mounted), also choosing adequate materials
%% for it. Based on the relevant requirements for the spacecraft and
%% its components, decide if the structure has to be designed for
%% stiffness and/or strength and/or internal pressure.

%% 6. Define the other mechanisms eventually needed by the spacecraft
%% (e.g. for deployment, pointing, separation, gimbal, etc.) and
%% select them among commercially existing components.
