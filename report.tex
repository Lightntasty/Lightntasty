%% Mission Objectives

%% Europa, one of the satellites of Jupiter, is probably the most
%% interesting celestial body in the Solar System due to its unique
%% characteristics, that have led some scientists to theorize a
%% scenario in which it could even be colonized in a far future. This
%% mission will therefore have the goal of better investigating the
%% surface characteristics and the potential habitability of Europa,
%% through a detailed observation of the ice-shell surface of the
%% planet and a characterization of the chemical composition of the
%% surface and the inner oceans below it. The spacecraft payload will
%% be represented by a high-resolution camera, a laser altimeter, an
%% ice-penetrating radar, a set of spectrometers and a thermal camera.

%% Top-Level Requirements
%% • Payload Mass (total): 80 kg
%% • Payload Size (total): 0.7 m x 0.7 m x 0.7 m
%% • Payload Required Power (maximum): 50 W
%% • Payload Operational Temperature Range: 150÷200 K
%% • Orbit: polar, 200 km altitude (above Europa surface)
%% • Mission Duration: at least 3 years in Europa orbit
%% • Ground Segment: ESTRACK
%% • Mission Operations Centre: ESOC
%% • Launch Date: 2020
%% • Total Mission Cost: ≤ 500 million Dollars (in FY2000 money)
%% • Mission Reliability: 0.9

\documentclass{report}
\usepackage[utf8]{inputenc}
\title{AE2100 \\
  Space Project 3 \\
  Europa Orbiter}
\author{Group C3 \\
  \begin{tabular}{rl}
    4145623 & Seong Hun Lee \\
    4160592 & Constantin Jux \\
    4191250 & Julius Jördens \\
    4165799 & Geart van Dam \\
    4160681 & Tim Blondeel \\
    4131959 & Aaku Kokko \\
\end{tabular}}
\date{}
\setcounter{secnumdepth}{-2}
\usepackage{amsmath}
\usepackage{cite}
\usepackage{siunitx}
\usepackage{longtable}
\usepackage{booktabs}
\usepackage{supertabular}
\usepackage{float}
\usepackage{graphicx}
\bibliographystyle{unsrt}
\newcommand{\workpkg}[1]{\part{Work Package #1}}
\newcommand{\subworkpkg}[1]{\chapter{WP #1}}
\newcommand{\deliverable}[1]{\section{D#1}}
\newcommand{\task}[1]{\subsection{Task #1}}
\usepackage{hyperref}
\hypersetup{
    colorlinks=false,
    pdfborder={0 0 0},
}

\input{matlab/data}
\begin{document}
\maketitle
\tableofcontents

\workpkg{2}

%% This Work Package deals with the detailed design of the spacecraft
%% subsystems, the revision of the mass and power budgets, and the
%% definition of the final spacecraft architecture.  WP 2.1

%% Timeline : Weeks 3, 4 and 5 (3-weeks total duration)

%% Deadline for Report Delivery: Monday October 8th, 12.00 hours

%% Each report shall include, as a minimum, all the Deliverables
%% listed in the sub-WP descriptions provided below (each deliverable
%% is identified by a unique code, in the form Dx.y.z). The names and
%% student numbers of the authors shall be present on the cover
%% page. The report shall also include a work division table,
%% indicating which group members have contributed to each one of the
%% tasks.  Deliver the printed report to the Teaching Assistant
%% assigned to your group.

\subworkpkg{2.1}

%% Deliverables

%% D2.1.1. For each one of the Tasks and sub-Tasks described above,
%% document in detail your choices, calculations and outcomes. Provide
%% detailed characteristics of the components you have chosen,
%% including their mass, size and power requirements; whereas
%% possible, include pictures and/or drawings of the components.

\deliverable{2.1.1}

%% Tasks

%% Note that all the Tasks below, related to different subsystems of
%% the spacecraft, need to be carried out in parallel, because several
%% key requirements and characteristics of the different subsystems
%% are significantly coupled each other. Hint: divide the whole group
%% into sub-groups (or individuals), each one working on a different
%% subsystem, and plan regular “system engineering meetings”
%% (e.g. daily) to exchange with the other sub-groups the information
%% related to the work done on each subsystem.

\task{A. Attitude Determination and Control System}
%% A. Attitude Determination and Control System (ADCS)

%% 1. Define the key subsystem requirements and functions (e.g.,
%% spacecraft stabilization and station keeping, precise orientation
%% of the payload instruments, orientation of the solar arrays, etc.).

%% 2. Evaluate whether different modes of operation of the ADCS are
%% needed or not.

%% 3. Evaluate the external and internal disturbances (forces and
%% torques) acting on the spacecraft.

%% 4. Define the type, number and size of attitude sensors and
%% attitude control actuators needed by the spacecraft (note: consider
%% sufficient redundancies!!)

%% 5. Select the sensors and actuators for the spacecraft among
%% commercially existing models, and dimension your subsystem based on
%% the characteristics of the selected components.

%% 6. Develop a block diagram that identifies the various functions
%% that need to be performed by the ADCS software.



\task{B. Propulsion}
%% B. Propulsion

%% 1. Define the key subsystem requirements and functions (e.g., orbit
%% insertion, orbital manoeuvres, interplanetary transfer,
%% etc.). Note: consider only the propulsion system on-board of the
%% orbiter spacecraft; design of optional kick stages (if any) is not
%% required.

%% 2. Define and quantify the eventual forces to be overcome by the
%% propulsion system (gravity, drag, solar radiation etc.).

%% 3. Revise your estimation for the total v required by the mission
%% and, based on it, select the thruster(s) and the propellants to be
%% used (note: decide at this stage if a “low thrust” or a “high
%% thrust” strategy has to be adopted, and select your thrusters
%% accordingly).

%% 4. Evaluate the total propellant mass needed by the mission. Don’t
%% forget to consider sufficient margins!!

%% 5. List and characterize all the components of the propulsion
%% systems; don’t consider the propellant tank(s), that will be
%% studied under the structures subsystem. If possible, select a
%% commercially available complete propulsion system.

%% 6. Generate a sketch showing the number and the location of the
%% various propulsion components in the spacecraft. Include in the
%% sketch also the major characteristics of the components shown.


\task{C. Thermal Control}
%% C. Thermal Control

%% 1. Define the key subsystem requirements and functions (e.g.,
%% operational temperatures of the spacecraft components, acceptable
%% thermal gradients, etc.).

%% 2. Define the thermal environment of the spacecraft (solar
%% intensity, planet flux, albedo, etc.). Focus should be on defining
%% the thermal environments that are most critical for the design,
%% i.e. those environments that lead to max/min temperature
%% conditions.

%% 3. Estimate the internal heat generated in the spacecraft.

%% 4. Generate a straw-man subsystem design using only passive means
%% of thermal control like paints, MLI (Multi-Layer Insulation) and
%% OSR (Optical Solar Reflector).

%% 5. Using the equations provided in Ref. [1], make a first
%% estimation of the maximum and minimum equilibrium temperature of
%% both the spacecraft body and the solar array (when present), in the
%% case when no thermal control systems are used.

%% 6. Based on your equilibrium temperature estimation and on the
%% temperature requirements for the different components, define a
%% thermal control strategy for the spacecraft and select (or design)
%% the appropriate subsystem components needed for it.  Where
%% possible, use commercially existing components or materials.


\task{D. Structures and Mechanisms}
%% D. Structures and Mechanisms

%% 1. Define the key requirements for the structures and mechanisms
%% subsystem.

%% 2. Define all the loads acting on the spacecraft during its
%% operations (note: take into

%% account the launch loads that you have already estimated during WP
%% 1!!). Define the safety factors to be used (you can use, for
%% instance, the values provided by the relevant ESA ECSS standards,
%% see Ref. [8]).

%% 3. Preliminarily dimension and design the primary structure of your
%% spacecraft, also choosing adequate materials for it. Based on the
%% relevant requirements for the spacecraft and its components, decide
%% if the structure has to be designed for stiffness and/or strength
%% and/or internal pressure. Hint: for your analysis you may consider
%% the primary structure to resemble a hollow cylindrical beam.

%% 4. Based on the amount of propellant(s) needed and the required
%% storage pressure, dimension and design the propellant tank(s), also
%% choosing adequate materials (note: the tank(s) can eventually be
%% used as primary structural elements!). Whereas possible, select
%% commercially existing tanks.

%% 5. Preliminarily dimension and design the supporting structure of
%% the solar arrays of your spacecraft (i.e., the mechanical structure
%% on which the array is mounted), also choosing adequate materials
%% for it. Based on the relevant requirements for the spacecraft and
%% its components, decide if the structure has to be designed for
%% stiffness and/or strength and/or internal pressure.

%% 6. Define the other mechanisms eventually needed by the spacecraft
%% (e.g. for deployment, pointing, separation, gimbal, etc.) and
%% select them among commercially existing components.


\task{E. Power}
%% E. Power

The minimum solar intensity $I$ at Jupiter occurs at the apohelion and is
\begin{equation}
  I_{\mathrm{min}} = I_{\si{AU}} (\frac{\SI{1}{AU} (1 - e)}{a})^2
  \approx \SI{45.7}{W/m^2}
\end{equation}

%% 1. Define the key power subsystem requirements and, in particular,
%% the total amount of power required by the spacecraft during all the
%% mission phases.

%% 2. Quantify the available solar energy, i.e. the solar intensity
%% and the direction of solar radiation, the duration of day time and
%% night time (eclipse) periods, the energy received from the planet.

%% 3. Make a preliminary dimensioning of the solar arrays and, in
%% particular, a first estimation of their size (note: detailed design
%% of the solar arrays will be done during WP 3!!). You should
%% consider, for instance, whether you would like a fixed array or an
%% array that can be rotated over one or two axes to allow for ideal
%% lighting conditions.

%% 4. Based on your power requirements, evaluate the main
%% characteristic properties of the spacecraft batteries (e.g.,
%% specific energy, energy density, capacity, mass, size, etc.).

%% 5. Select the batteries and the power control components among
%% commercially existing models.

%% 6. Generate a block diagram showing the main subsystem components,
%% the power flow and the efficiencies of power conversion.


\task{F. Payload}
%% F. Payload

%% 1. Based on the general payload requirements and instruments list
%% provided in the project description of Section 3, choose the
%% payload components to be used by your spacecraft among commercially
%% available or previously used/proposed ones. Hint: for accomplishing
%% this task, you can also make use of the information included in
%% Ref. [7].

%% 2. Define the positioning and interface constraints for all the
%% payload components.


\task{G. Other Subsystems}
%% G. Other Subsystems

%% 1. For the remaining subsystems (command and data handling,
%% communications, navigation, etc.) define their key requirements and
%% functions and preliminarily select their main components. Don’t go
%% too much into the details for these subsystems; just analyse their
%% general architecture and their impact on the design of the other
%% spacecraft subsystems.


\subworkpkg{2.2}

%% Tasks

%% A. Using the result obtained by WP 2.1 (in particular, the
%% requirements and characteristics of the components chosen for the
%% subsystems), define detailed mass and power budgets for the
%% spacecraft.  Clearly indicate the margins that are available.

%% B. Compare the new budgets to the preliminary ones defined in the
%% framework of WP 1, and highlight eventual changes or significant
%% differences.

%% C. Verify that the mass and power budgets are compliant to ALL the
%% mission requirements and, if necessary, re-iterate the subsystems
%% design obtained during WP 2.1.

%% Deliverables

%% D2.2.1. Tables showing the detailed mass and power budgets for the
%% spacecraft.

\deliverable{2.2.1}

\subworkpkg{2.3}

%% Tasks

%% A. Based on the operational requirements, the interfaces and the
%% dimensions of the subsystems and their components, define a
%% detailed architecture of the spacecraft including all the relevant
%% subsystems.

%% B. Draft your detailed spacecraft architecture as a CATIA drawing,
%% and compare it to the preliminary one(s) obtained in the framework
%% of WP 1.

%% Deliverables

%% D2.3.1. CATIA drawing of the spacecraft architecture. The drawing
%% shall include one or more 3D views and the necessary 2D
%% views/cut-offs needed to show the position and interfaces of all
%% subsystems.

\deliverable{2.3.1}


\chapter{Division of tasks.}

\begin{longtable}{rcccccc}
  \caption{Division of tasks between the group members. E: editor. A: author.} \\
  & Seong Hun Lee & Jux & Jördens & van Dam & Blondeel & Kokko \\
\end{longtable}

\bibliography{ae2100}

\end{document}
