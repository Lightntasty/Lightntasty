%% C. Thermal Control

%% 1. Define the key subsystem requirements and functions (e.g.,
%% operational temperatures of the spacecraft components, acceptable
%% thermal gradients, etc.).

%% 2. Define the thermal environment of the spacecraft (solar
%% intensity, planet flux, albedo, etc.). Focus should be on defining
%% the thermal environments that are most critical for the design,
%% i.e. those environments that lead to max/min temperature
%% conditions.

%% 3. Estimate the internal heat generated in the spacecraft.

%% 4. Generate a straw-man subsystem design using only passive means
%% of thermal control like paints, MLI (Multi-Layer Insulation) and
%% OSR (Optical Solar Reflector).

%% 5. Using the equations provided in Ref. [1], make a first
%% estimation of the maximum and minimum equilibrium temperature of
%% both the spacecraft body and the solar array (when present), in the
%% case when no thermal control systems are used.

%% 6. Based on your equilibrium temperature estimation and on the
%% temperature requirements for the different components, define a
%% thermal control strategy for the spacecraft and select (or design)
%% the appropriate subsystem components needed for it.  Where
%% possible, use commercially existing components or materials.
